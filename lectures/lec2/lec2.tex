\documentclass[../../main/main.tex]{subfiles}

\begin{document}
\chapter{24/09/2020}
\label{cpt:lec2}

\section{Euclidean Topology in $\mathbb{R}$}
\label{sec:euclidean-topology}

\begin{definition}
  A set $A \subseteq \mathbb{R}$ is open in the Euclidean topology if $\forall a \in A, \exists b, c \in \mathbb{R}$ such that $a \in (b, c) \subseteq A$.
\end{definition}

To see that this is a topology on $\mathbb{R}$ notice that:

\begin{enumerate}
  \item $\mathbb{R}, \emptyset \in \Top$
  \item Let $A, B$ be open sets and $a$ an element of $A \cap B$. Then there exist $b_{1}, c_{1} \in A$ and $b_{2}, c_{2} \in B$ such that $a \in (b_{1}, c_{1}) \subseteq A$ and $a \in (b_{2}, c_{2}) \subseteq B$. Thus $a \in \left( \max (b_{1}, b_{2}), \min (c_{1}, c_{2}) \right) \subseteq A \cap B$.
  \item Let $A_{i} \in \Top$ for $i \in I$. Then $a \in \bigcup_{i \in I} A_{i} \Rightarrow a \in A_{j}$ for some $j \in I$. As such, there exist $b, c \in A_{j}$ such that $a \in \left( b, c \right) \subseteq A_{j} \subseteq \bigcup_{i \in I} A_{i}$.
\end{enumerate}

In this topology we have that, for $a < b \in \mathbb{R}$

\begin{itemize}
  \item $\left( a, b \right)$ is an open
  \item $\left( -\infty, a \right) $  and $\left( a, \infty \right) $ are open
  \item $\left[ a, b \right] $ is closed, because
    \begin{equation*}
      \mathbb{R} \setminus \left[ a, b \right] = \left( -\infty, a \right) \cap \left( b, \infty \right)  \in \Top
    \end{equation*}
  \item $ ( - \infty, a ]$ and $[a, \infty)$ are closed
  \item Any singular subset $\left\{ a \right\} $ is closed
  \item $ \left[ a, b \right] $ is not open.
    \begin{proof}
      Supposed $\left[ a, b \right] $ is open and let $c \in [a, b]$. Then there exist $d, e \in [a, b]$ such that $c \in (d, e)$. Taking $c = a$, we analyse the lower bound
      \begin{equation*}
        a \leq  d < c =a \Longrightarrow a < a
      \end{equation*}
      which constitutes a contradiction. The same could be done for the upper bound. In this way, any finitely bounded closed interval is not open.
    \end{proof}
  \item $\mathbb{Z}$ is closed but not open.
  \item $\mathbb{Q}$ is closed but not open.
\end{itemize}

\begin{theorem}
  Let $A \subseteq \mathbb{R}$. Then A is open in the Euclidean topology if and only if it is an union of open sets.
\end{theorem}
\begin{proof}
  Assume $A$ is open. Then $\forall a \in A$ there exists $(b_{a}, c_{a}) \subseteq A$ such that $a \in (b_{a}, c_{a})$. As such
  \begin{equation*}
    A = \bigcup_{a \in A} (b_{a}, c_{a}).
  \end{equation*}
  Now assume A is the union of open sets. Then A is open too.
\end{proof}

\begin{definition}
  Let $(X, \Top)$ be a topological space. Then $B \subseteq \Top$ is a base of $\Tau$ if any open set is an union of members of $B$.
\end{definition}

\begin{example}
  The open sets constitute a base of the Euclidean topology.
\end{example}

\begin{example}
  The singular sets constitute a base of the discrete topology.
\end{example}

\begin{definition}
  A family of subsets $B \in \mathcal{P}(X)$ constitutes a base of a topology in $X$ if and only if $X$ is an union of members of $B$, and the intersection of members of $B$ is in $B$.
\end{definition}

\begin{example}
  Let $X = \{a, b, c\}$ and $B = \{ \{a\}, \{b\}\}$. Then B is not the base of any topology on $X$ because $X \in \Top$ for any topology, but $X$ is not generated by $B$.
\end{example}

\end{document}
